% Document Settings
\documentclass[12pt, a4paper, oneside]{extbook}
%\usepackage[utf8]{inputenc}
\usepackage[T1]{fontenc}

\usepackage[
	top=2cm,
	bottom=2cm,
	left=2cm,
	right=2cm,
	headheight=17pt,
	includehead, includefoot,
	footskip=50pt
]{geometry} 


% AMS Packages
\usepackage{amsmath}
\usepackage{amsfonts}
\usepackage{amssymb}



% graphicx
\usepackage{graphicx}
\graphicspath{{../images/}{../images/part1}}

\newcommand{\img}[1]{ % for inline image insertion
	\ %
	\raisebox{-0.1\baselineskip}{%
		\includegraphics[
		height=\baselineskip,
		width=\baselineskip,
		keepaspectratio,
		]{#1}%
	}%
	\ %
}

\newcommand{\tabimg}[1]{ % for in-tabular image insertion
	\raisebox{-0.3ex}{%
		\includegraphics[
		height=1.2em,
		keepaspectratio,
		]{#1}%
	}%
}


% Paragraph Style
\usepackage{indentfirst}
\setlength{\parindent}{2em}
\renewcommand{\baselinestretch}{1.25}
\setlength{\parskip}{1ex}

% Fonts
\usepackage{fontspec}
\usepackage{xeCJK}
\setCJKfamilyfont{ming}{新細明體}
\setCJKfamilyfont{kai}{標楷體}
\setCJKfamilyfont{hei}{Taipei Sans TC Beta}
\setCJKmainfont[AutoFakeBold=1.5]{標楷體}
\setCJKsansfont{Noto Sans CJK TC}
%\setCJKsansfont{Taipei Sans TC Beta}
\setCJKmonofont{Noto Sans CJK TC}

%\setmainfont{Times New Roman}
\setsansfont{Aller}
\setmonofont{Consolas}

% For TeX-related logos
\usepackage{hologo}
\hologoFontSetup{general=\rmfamily}
\let\tmpTeX\TeX
\renewcommand{\TeX}{\textrm{\tmpTeX}}
\let\tmpLaTeX\LaTeX
\renewcommand{\LaTeX}{\textrm{\tmpLaTeX}}

% Floats, Figures, Tables, ...
\usepackage[table]{xcolor}
\usepackage{float}
\usepackage[justification=centering, labelfont=sf, textfont=sf]{caption}


\renewcommand{\figurename}{圖}

\usepackage{array}
\usepackage{tabularx}
\usepackage{booktabs}
\usepackage{multirow}
\usepackage{diagbox}
\renewcommand{\tablename}{表}

\setlength{\arrayrulewidth}{1pt}
\arrayrulecolor{black}
\renewcommand{\arraystretch}{1.5}

\newcolumntype{L}{>{\raggedright\arraybackslash}X}
\newcolumntype{C}{>{\centering\arraybackslash}X}
\newcolumntype{R}{>{\raggedleft\arraybackslash}X}


\usepackage{makecell}
\renewcommand{\theadfont}{\centering\bfseries\large}
\newcommand{\Thline}{\Xhline{2\arrayrulewidth}}
\newcolumntype{T}{!{\vrule width 2\arrayrulewidth}}
%\newcolumntype{H}[1]{>{\hsize=#1\hsize\arraybackslash}X}



% Enumerate and Itemize
\usepackage{enumitem}
\setlist{
	labelsep=1ex,
	labelindent=\parindent,
	nosep
}


% Enum and Item in tabular envs (to avoid weird spacing issues)
\newenvironment{tabitemize}{%
	\begin{minipage}[t]{\linewidth}
		\begin{itemize}[label=\textbullet, nosep, left=0pt, after={\vspace{0.5\baselineskip}}]
}{%
		\end{itemize}
	\end{minipage}
}
% Figures in tabular envs
\newcommand{\tabfig}[2][0.95]{%
	\resizebox{#1\linewidth}{!}{#2}
}

\newenvironment{tabmp}{%
	\begin{minipage}[t]{\linewidth}
%		\centering
		\vspace{-0.5\baselineskip}%
}{%
		\vspace{0.5\baselineskip}
	\end{minipage}
}

% Code
\usepackage[chapter]{minted}
\setminted{
	breaklines,
	autogobble,
	tabsize=4,
}


% tcolorbox
\usepackage[breakable, minted, skins]{tcolorbox}

%
\renewcommand{\listingscaption}{程式碼}
\renewcommand{\theFancyVerbLine}{\ttfamily{\footnotesize{\arabic{FancyVerbLine}}}} % redefine style of line numbers
\newtcblisting{src}{
	listing only,
	breakable,
	enhanced,
	listing engine=minted,
	minted language=LaTeX,
%	minted style=colorful,
	minted options = {
		linenos, 
		breaklines=true,
		breakanywhere, 
		breakanywheresymbolpre=,
		numbersep=3.75mm,
		autogobble,
		tabsize=4
	},
	drop fuzzy shadow,
	left=7mm,
	arc=2pt,
	overlay = {
		\begin{tcbclipinterior}
			\fill[gray!25] (frame.south west) rectangle ([xshift=7mm]frame.north west);
		\end{tcbclipinterior}
	}  
}



% hyperref
\usepackage{hyperref}
\hypersetup{
	colorlinks=true,
	linkcolor=black,
	filecolor=magenta,   
	urlcolor=blue
}

% Header
\usepackage{fancyhdr}
\fancypagestyle{plain}{
	\renewcommand{\headrulewidth}{0pt}
	\renewcommand{\footrulewidth}{0pt}
	\fancyhf{}
	\rhead{\ttfamily\thepage}
%	\rfoot{\includegraphics[width=10em]{/logo/\arabic{page}.jpg}}
}
\fancypagestyle{fancy}{
	\renewcommand{\headrulewidth}{2pt}
	\renewcommand{\footrulewidth}{2pt}
	\fancyhf{}
	\rhead{\ttfamily\thepage}
	\lhead{\leftmark}
%	\cfoot{\leftmark}
	\rfoot{\includegraphics[width=5em]{/logo/\arabic{page}.jpg}}
}
\pagestyle{fancy}




% Chemistry and Physics
\usepackage{chemfig}
\renewcommand\printatom[1]{\ensuremath{\mathsf{#1}}}

\usepackage{tikz}
\usepackage[american]{circuitikz}

%
\usepackage{texmate}


%zhnumber
\usepackage{zhnumber}


% Counters
\usepackage{chngcntr}
\counterwithin{chapter}{part}
\renewcommand{\thechapter}{\arabic{chapter}}


% Title Styles
\usepackage{titlesec}
\titleformat{\chapter}[frame]{\normalfont\huge\bfseries}{第\ \thechapter\ 章}{20pt}{\Huge}
\titlespacing{\chapter}{0pt}{-10pt}{40pt}
\titleformat{\section}
{\normalfont\Large\bfseries}{\thesection}{1em}{}
%\titleformat{\part}[]


% Debug
%\usepackage{lipsum}
%\usepackage{showframe}

% UNUSED
%\usepackage{moreenum}



% TOC
\renewcommand{\contentsname}{目錄 Table Of Contents}





\begin{document}
%	\maketitle
	\begin{titlepage}
		\begin{center}
			\sffamily
			
%			\vspace*{1cm}
%			temp title design
			\begin{tcolorbox}[
				halign=flush center,
				drop fuzzy shadow,
				arc=2pt,
				height=4cm,
				valign=center]
				\fontsize{36pt}{\baselineskip}\selectfont \LaTeX\ 祕笈
			\end{tcolorbox}
			
			\vspace*{1cm}
		
			{\fontsize{24pt}{\baselineskip}\selectfont 武陵資訊讀書會 - \fontspec{Torus}{} GPwaob\_92679}
			
			\vfill
			
%			temp image
			\includegraphics[width=0.8\textwidth]{readme/science.png}
		
		
		\end{center}
	\end{titlepage}

	\tableofcontents

	\part{前言}
	\chapter*{作者介紹}
	\addcontentsline{toc}{chapter}{作者介紹}
	\chapter*{推薦序}
	\addcontentsline{toc}{chapter}{推薦序}  
	
	
	
	
	\chapter{\LaTeX 簡介}
		歡迎各位讀者進入\LaTeX 的世界!
		
		
%		你現在正在閱讀的這份祕笈就是用\LaTeX 排版出來的喔!

%		情境
%		故事 時間線
	\section{什麼是\TeX ?}
		\TeX 是美國電腦科學家Donald Knuth在1978年發表的一套排版軟體,。相較於市面上大多的排版軟體(如: Microsoft Word、LibreOffice Writer、Google Docs),\TeX 沒有漂亮的圖形化介面(GUI),而是像寫程式一樣,先把指令(告訴電腦東西應該長怎樣)與文件內容(真的給人看的東西)寫在一個純文字檔後,再經過編譯器的編譯,產生最後供人類閱的文件檔。
	\section{什麼是\LaTeX ?}
		隨著科技的發展,當年的\TeX 所提供的功能早已不敷使用,同時也被人覺得太複雜、不親民。因此,美國又有一位電腦科學家Leslie Lamport在1984年發表了基於\TeX 的排版系統\LaTeX ,提供更多實用的功能和更親民的指令集,也推廣了這套系統的應用。
		
		講白話一點,\LaTeX 其實就是\TeX 的PRO版,而且比原本的\TeX 好用n百倍,導致\TeX 被大眾打入冷宮了。

	\section{為什麼要學\LaTeX ?}
		\LaTeX 被廣泛應用於數學公式與科學論文的排版,許多學術期刊及國外的教科書都是使用\LaTeX 排版出。
	\section{\LaTeX 究竟有多強大?}
		說到\LaTeX ,第一個想到的當然是它在數學公式與科學圖表優秀的排版啦!
		\begin{table}[H]
			\centering
			\rowcolors{2}{gray!20}{gray!10}
			\setminted{fontsize=\tiny, breakafter=-=\}, breakaftersymbolpre=}
			\begin{tabular}{Tp{5em}|>{\small\ttfamily}p{12ex}|>{\centering\arraybackslash\sffamily}p{5.5cm}|p{6cm}T}
				\Thline 
				\rowcolor{cyan!40} \thead{應用} & \thead{\textrm{套件}} & \thead{\textrm{範例}} & \thead{原始碼}(部分省略) \\ \hline
				數學公式 & amsmath &  & \\ \hline
				化學結構式 & chemfig &
				\begin{tabmp}
					\centering
					\tabfig{\chemname[3.5ex]{\chemfig[angle increment=30]{*6(-=(--[1]-[-1]-[1]-[-1])-=(-OH)-(*6(-(<:[1]H)(*6(-=(-)---))-(<[7]H)-(-[6])(-[8])-O-))=)}}{\Large 四氫大麻酚(Tetrahydrocannabinol, THC)}}
				\end{tabmp} &
				\begin{tabmp}
					\begin{minted}{LaTeX}
						\chemname[3ex]{\chemfig[angle increment=30]{
						*6(-=(--[1]-[-1]-[1]-[-1])-=(-OH)-(*6(-(<:[1]H)
						(*6(-=(-)---))-(<[7]H)-(-[6])(-[8])-O-))=)}
						}{四氫大麻酚(Tetrahydrocannabinol, THC)}
					\end{minted}
				\end{tabmp} \\ \hline
				電路圖 & circuitikz &
				\begin{tabmp}
					\centering
					\tabfig{
						\begin{circuitikz}
							\fill[blue!15!white] (-1, 0.8) rectangle (0.5,-0.8);
							\fill[orange!20!white] (-1, -1.2) rectangle (0.5, -2.8);
							\node[draw, color=blue] at (1.7, 0){\textbf{P-channel}};
							\node[draw, color=orange] at (1.7, -2){\textbf{N-channel}};
							\ctikzset{tripoles/pmos style/emptycircle}						
							\draw (0,0) node[pmos](P){};
							\draw (0, -2) node[nmos](N){};
							\draw (P.D) -- (N.D);
							\draw (P.S) to[short, -*] ++(0, 0.5) node[above]{$V_{dd}$};
							\draw (N.S) -- ++(0, -0.5) node[ground](GND){}
							(GND.south) node[below]{$GND$};
							\draw (P.G) -- ++(-1, 0) -- ++(0, -1) node[](in){} -- ++(0, -1) -- (N.G);
							\draw (in.center) to[short, *-*] ++(-1, 0) node[left]{$V_{in}$};
							\draw (0, -1) to[short, *-*] ++(1, 0) node[right]{$V_{out}$};
						\end{circuitikz}
					}
				\end{tabmp} &
				\begin{tabmp}
					\begin{minted}{LaTeX}
						\begin{circuitikz}
							\fill[blue!15!white] (-1, 0.8) rectangle (0.5,-0.8);
							\fill[orange!20!white] (-1, -1.2) rectangle (0.5, -2.8);
							\node[draw, color=blue] at (1.7, 0){\textbf{P-channel}};
							\node[draw, color=orange] at (1.7, -2){\textbf{N-channel}};
							\draw (0,0) node[pmos](P){};
							\draw (0, -2) node[nmos](N){};
							\draw (P.D) -- (N.D);
							\draw (P.S) to[short, -*] ++(0, 0.5) node[above]{$V_{dd}$};
							\draw (N.S) -- ++(0, -0.5) node[ground](GND){}
							(GND.south) node[below]{$GND$};
							\draw (P.G) -- ++(-1, 0) -- ++(0, -1) node[](in){} -- ++(0, -1) -- (N.G);
							\draw (in.center) to[short, *-*] ++(-1, 0) node[left]{$V_{in}$};
							\draw (0, -1) to[short, *-*] ++(1, 0) node[right]{$V_{out}$};
						\end{circuitikz}
					\end{minted}
				\end{tabmp} \\ \Thline
			\end{tabular}
			\label{tab: Scientific Applications of LaTeX}	
		\end{table}
		但是這麼強大的軟體,不拿來做一些趣味用途真是太可惜了!
		\begin{table}[H]
			\centering
			\rowcolors{2}{gray!20}{gray!10}
			\setminted{fontsize=\tiny, breakafter=-=\}, breakaftersymbolpre=}
			\begin{tabular}{Tp{5em}|>{\small\ttfamily}p{12ex}|>{\centering\arraybackslash\sffamily}p{5.2cm}|p{6cm}T}
				\Thline 
				\rowcolor{cyan!40} \thead{應用} & \thead{\textrm{套件}} & \thead{\textrm{範例}} & \thead{原始碼}(部分省略) \\ \hline
				西洋棋 & skak \newline texmate &
				\begin{tabmp}
					\vspace{-0.8\baselineskip}
					\renewcommand{\afterno}{.}
					\whitename{Adolf Anderssen}
					\blackname{Jean Dufresne}
					\chessevent{Berlin/Berlin GER/1852}
					\chessopening{Evans Gambit}
					\ECO{C52}
					\tiny
					\makegametitle
					\begin{texmate}
						1.e4 e5 2.Nf3 Nc6 3.Bc4 Bc5 4.b4 Bxb4 5.c3 Ba5 6.d4 exd4 7.O-O d3 8.Qb3 Qf6 9.e5 Qg6 10.Re1 Nge7 11.Ba3 b5 12.Qxb5 Rb8 13.Qa4 Bb6 14.Nbd2 Bb7 15.Ne4 Qf5 16.Bxd3 Qh5 17.Nf6+ gxf6 18.exf6 Rg8 19.Rad1 Qxf3 20.Rxe7+ Nxe7 21.Qxd7+ Kxd7 22.Bf5+ Ke8 23.Bd7+ Kf8 24.Bxe7\# \result{1-0}
					\end{texmate}
					\vspace{1mm}
					\smallboard
					\notationon
					\preparediagram{Evergreen Game}{(Final position after 24.Bxe7\#)}
					\centering
					\tabfig[0.8]{\makediagrams}
				\end{tabmp} &
				\begin{tabmp}
					\begin{minted}{LaTeX}
						\whitename{Adolf Anderssen}
						\blackname{Jean Dufresne}
						\chessevent{Berlin/Berlin GER/1852}
						\chessopening{Evans Gambit}
						\ECO{C52}
						\makegametitle
						
						\begin{texmate}
							1.e4 e5 2.Nf3 Nc6 3.Bc4 Bc5 4.b4 Bxb4 5.c3 Ba5 6.d4 exd4 7.O-O d3 8.Qb3 Qf6 9.e5 Qg6 10.Re1 Nge7 11.Ba3 b5 12.Qxb5 Rb8 13.Qa4 Bb6 14.Nbd2 Bb7 15.Ne4 Qf5 16.Bxd3 Qh5 17.Nf6+ gxf6 18.exf6 Rg8 19.Rad1 Qxf3 20.Rxe7+ Nxe7 21.Qxd7+ Kxd7 22.Bf5+ Ke8 23.Bd7+ Kf8 24.Bxe7\# \result{1-0}
						\end{texmate}
						\notationon
						\preparediagram{Evergreen Game}{(Final position after 24.Bxe7#)}
						\makediagrams
					\end{minted}
				\end{tabmp} \\ \hline
				象棋 & cchess &
				\begin{tabmp}
					\tabfig{
					}
				\end{tabmp} &
				\begin{tabmp}
					\begin{minted}{LaTeX}
					\end{minted}
				\end{tabmp} \\ \hline
				五線譜 & musixtex &
				&
				\\ \Thline
			\end{tabular}
			\label{tab: Fun Applications of LaTeX}	
		\end{table}
	\section{\LaTeX 的優缺點?}
		優點:
		\begin{itemize}
			\item 完全免費。
			\item 編輯檔案時不受作業系統限制,只要是打得出字的機器都能直接編輯。
			\item 非常適合數學公式的排版,可以排出漂亮的報告與論文。
			\item 
		\end{itemize}
		
		缺點:
		\begin{itemize}
			\item 嵌入圖片與表格很麻煩,格式很難調。
			\item 初始設定複雜、指令記憶難,入坑門檻高。
			\item 
			\item 
		\end{itemize}

	\section{\LaTeX 與其他軟體的比較}
		首先,當然是與市面上最普遍的文書處理軟體Word來個大比拚啦!
		
		\begin{table}[H]
			\centering
			\rowcolors{2}{gray!20}{gray!10}
			\begin{tabularx}{0.8\textwidth}{T X | X T}
				\Thline
				\rowcolor{cyan!50} \thead{\LaTeX \ LaTeX} &
				\thead{\tabimg{ms_word.png} Microsoft Word} \\ \hline
				aaa & aaa \\ \hline
				bbb & bbb \\ \hline
				ccc & ccc \\ \hline
				ddd & ddd \\ \Thline
			\end{tabularx}
			
% insert microsoft word meme
% 程式碼 vs WYSIWYG
			\caption{\LaTeX\ vs. Microsoft Word}
			\label{tab:LaTeX vs MSWord}
		\end{table}
		
		再者,既然說\LaTeX 比較像是一種程式語言,那當然要拿來與時下當紅的C++與Python來比較啊!
		
		\begin{table}[h]
			\centering
			\rowcolors{2}{gray!20}{gray!10}
			\sffamily
			\begin{tabularx}{\textwidth}{Tp{0.26\textwidth}TX|X|XT}
				\Thline
				\rowcolor{cyan!50} \diagbox[innerwidth=0.26\textwidth, linewidth=2\arrayrulewidth]{使用軟體/屬性}{語言} &
				\thead{\LaTeX \ \ LaTeX} &
				\thead{\tabimg{cpp_logo.png}\ \ C++} &
				\thead{\tabimg{python-logo.png}\ \ Python} \\ \Thline
				編輯器(Editor)\ / \newline
				整合開發環境(IDE\footnotemark[1]) &
				\begin{tabitemize}
					\item {Texmaker \tabimg{texmaker_logo.png}}
					\item {TeXstudio \tabimg{Texstudio_Logo.png}}
					\item {TeXworks \tabimg{TeXworks256.png}}
				\end{tabitemize} &
				\begin{tabitemize}
					\item {Dev-C++ \tabimg{Dev_C++_logo.png}}
					\item {Code::Blocks \tabimg{CodeBlocks_logo.png}}
				\end{tabitemize} &
				\begin{tabitemize}
					\item {PyCharm \tabimg{PyCharm_Icon.png}}
					\item {Spyder \tabimg{Spyder_logo.png}}
					\item {Jupyter \tabimg{Jupyter_logo.png}}
				\end{tabitemize} \\ \hline
				編譯器(Compiler)\ / \newline
				直譯器(Interpreter) &
				\begin{tabitemize}
					\item \hologo{pdfLaTeX}
					\item \hologo{XeLaTeX}
					\item \hologo{LuaLaTeX}
				\end{tabitemize} & 
				\begin{tabitemize}
					\item {G++ \tabimg{GNU_Compiler_Collection_logo}}
					\item {MSVC\footnotemark[2] \tabimg{Visual_Studio_Icon_2019.png}}
				\end{tabitemize} &
				\begin{tabitemize}
					\item CPython
					\item PyPy
				\end{tabitemize} \\ \hline
				輸出(Output) & .pdf \tabimg{PDF_file_icon.png} & {.exe \tabimg{exe_icon.png}} & N/A \\ \hline
				&  & & \\
				\Thline
			\end{tabularx}
			\caption{\LaTeX\ vs\ C++\ vs\ Python}
			\label{tab:LaTeX vs C++ vs Python}
		\end{table}
		\footnotetext[1]{Integrated Development Environment}
		\footnotetext[2]{Microsoft Visual C++}
		
	\chapter{環境初始設定}
		\LaTeX 並非一個獨立運行的程式,而是仰賴許多其他套件包(Packages)與...運行的排版系統。
		
	\section{\LaTeX 的編譯流程}
		在開始安裝
	\section{安裝\LaTeX 發行版 \textemdash\ \hologo{MiKTeX}}
	
	\section{安裝\LaTeX 編輯器 \textemdash\ Texmaker}

	\section{設定Texmaker}
		在此筆者使用\hologo{XeLaTeX}作為編譯引擎,因為他對多國語言的支援度相當廣泛,也是所有引擎中對中文支援最好的引擎之一。
	
	\part{\LaTeX 入門}
			
	\chapter{寫出你的第一份 \LaTeX 文件 !}
	\section{Hello, World! in \LaTeX}

	\begin{listing}[h]
		\begin{src}
			\documentclass{article}
			\begin{document}
				Hello, World!
			\end{document}
		\end{src}
		\caption{Hello World! in \LaTeX}
		\label{lst:Hello_World}
	\end{listing}
		

	
\end{document}
\documentclass[preview, border=1cm]{standalone}

\usepackage{float}
\usepackage[justification=centering, labelfont=sf, textfont=sf]{caption}

\usepackage{indentfirst}
\setlength{\parindent}{2em}
\renewcommand{\baselinestretch}{1.25}
\setlength{\parskip}{1ex}

\usepackage{fontspec}
\usepackage{xeCJK}
\setCJKmainfont{標楷體}
\setCJKsansfont{Noto Sans CJK TC}
\setsansfont{Aller}
\setmonofont{Consolas}

\renewcommand{\familydefault}{\sfdefault}

\renewcommand{\figurename}{圖}
\renewcommand{\tablename}{表}

\usepackage[table]{xcolor}
\usepackage[american]{circuitikz}
\usepackage{caption}

\begin{document}

	\begin{figure}[h]
		\centering
		
		\begin{circuitikz}
			\fill[blue!15!white] (-1, 0.8) rectangle (0.5,-0.8);
			\fill[orange!20!white] (-1, -1.2) rectangle (0.5, -2.8);
			
			\node[draw, color=blue] at (1.7, 0){\textbf{P-channel}};
			\node[draw, color=orange] at (1.7, -2){\textbf{N-channel}};
			
			\ctikzset{tripoles/pmos style/emptycircle}
			\draw (0,0) node[pmos](P){};
			\draw (0, -2) node[nmos](N){};
			\draw (P.D) -- (N.D);
			\draw (P.S) to[short, -*] ++(0, 0.5) node[above]{$V_{dd}$};
			\draw (N.S) -- ++(0, -0.5) node[ground](GND){}
			(GND.south) node[below]{$GND$};
			\draw (P.G) -- ++(-1, 0) -- ++(0, -1) node[](in){} -- ++(0, -1) -- (N.G);
			\draw (in.center) to[short, *-*] ++(-1, 0) node[left]{$V_{in}$};
			\draw (0, -1) to[short, *-*] ++(1, 0) node[right]{$V_{out}$};
		\end{circuitikz}
	
		\label{fig:CMOS}
		\caption[CMOS]{靜態互補式金屬氧化物半導體反相器(Static CMOS inverter) \\
			註:CMOS(互補式金屬氧化物半導體, Complementary Metal-Oxide-Semiconductor)}
	\end{figure}

	\arrayrulecolor{white}
	\setlength{\arrayrulewidth}{1pt}
	
	\begin{table}[h]
		\centering
		\begin{tabular}{|c|c|}
			\hline
			\rowcolor{cyan!50} Input & Output \\ \hline
			\rowcolor{cyan!20} L     & H      \\ \hline
			\rowcolor{cyan!20} H     & L      \\ \hline
		\end{tabular}
		
		\label{tab:CMOS_LH}
		\caption{CMOS反相器的真假值表}
	\end{table}

\end{document}